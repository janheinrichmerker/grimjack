\section{Conclusion}

To retrieve relevant argumentative passages, we combine query reformulation and expansion techniques with axiomatic re-ranking exploiting argumentative structure, quality, and stance.
Using the IBM Debater API and the T0++~language model, we showcase two state-of-the-art approaches for argument quality tagging.
We extend previous query expansion approaches from the Touché shared tasks by incorporating the contextual information provided in topic descriptions and narratives.
To attain nearly equal exposure across argument stances in the final ranking, we balance the pro and con arguments on top-10 ranks.

While none of our runs can outperform the BM25 baseline in terms of nDCG@5 on relevance and quality judgments, we find that axiomatic re-ranking and stance-based re-ranking for comparative arguments can slightly increase nDCG@5 effectiveness compared to our query likelihood baseline ranking. This poses an interesting direction of future research: Can retrieval effectiveness in comparative question answering be improved by alternating or balancing object stances~(e.g., in the BM25 baseline or other participant approaches)?
Because our run featuring query expansion with generated texts by the T0++ language model is the worst-performing of all our submitted runs in terms of relevance and rhetorical quality, we also question the usefulness of large language models in early retrieval stages. Our results represent additional motivation to investigate the effect of explainability on retrieval performance, as recently questioned in the community.

Our approach for stance classification is based on single-target stance classification, and we did not find a natural way to distinguish neutral arguments from arguments without stance while aggregating the objects' single-target stances to form a multi-target stance.
Arguably, fine-tuning a multi-class neural classifier like \Bert on the stance dataset provided by \citeauthor{BondarenkoFKSGBPBSWPH2022} could possibly improve classification performance by directly predicting the multi-target stance.